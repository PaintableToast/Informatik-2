% preamble.tex aus der Vorlesung
\documentclass[a4paper]{article}
\usepackage[a4paper,left=1.5cm,right=1.5cm,top=2cm,bottom=3cm]{geometry}
\usepackage{etex}
\usepackage[utf8]{inputenc}
\usepackage[T1]{fontenc}
\usepackage{lmodern}
\usepackage{tabto}
\usepackage{hyphsubst}
\HyphSubstIfExists{ngerman-x-latest}{%
  \HyphSubstLet{ngerman}{ngerman-x-latest}}{}
\HyphSubstIfExists{german-x-latest}{%
  \HyphSubstLet{german}{german-x-latest}}{}
\usepackage[english,ngerman]{babel}
\usepackage{textcomp}
\usepackage{microtype}

% Figuren Package
\usepackage{tikz}
\usepackage{pgfplots}
\pgfplotsset{compat=1.17}

%Math Packages
\usepackage{nicefrac,mathtools,enumitem}
\usepackage[all]{xy}
\usepackage[lite]{amsrefs}
\usepackage{amsmath}
\usepackage{amsthm}
\usepackage{amsfonts}
\usepackage{amssymb}
\usepackage{mathtools}

\newtheorem{definition}[equation]{Definition}
\newtheorem{satz}[equation]{Satz}

% Math Commands
\newcommand*{\ksum}[2][n]{\sum^{#1}_{k = #2}}
\newcommand*{\norm}[1][x]{\|#1\|}
\newcommand*{\elemof}[2][n]{#1 \in \mathbb{#2} } % Element of (N, Z, Q, R, C)
\newcommand*{\R}{\mathbb{R}} % R
\newcommand*{\K}{\mathbb{K}} % K
\newcommand*{\C}{\mathbb{C}} % C
\newcommand*{\N}{\mathbb{N}} % N

\newcommand{\complex}[2]{|#1|(\cos (#2) + i \sin (#2))}
\newcommand*{\Rn}{$\R^n$ } % R^n als Text Print


\title{Vorlesung 1}
\author{Ramon Cemil Kimyon}
\date{}


\begin{document}
\section{Haskell}
\subsection{Haskell Typen}
\[
\begin{array}{ll}
    \textsf{Float} & \textnormal{Gleitkomma-Zahlen mit einfacher Genauigkeit 32bit} \\
    \textsf{Double} & \textnormal{Gleitkomma-Zahlen mit doppelter Genauigkeit 64bit} \\
    \textsf{Int} & \textnormal{beschränktes Ganzzahl} \\
    \textsf{Integer} & \textnormal{unbeschränkte Ganzzahl} \\
    \textsf{Bool} & \textnormal{Wahrheitswerte true/false} \\
    \textsf{Char} & \textnormal{Zeichen} \\
    \textsf{Type} & \textnormal{Liste von Typen} \\
    \textsf{(TypeA, TypeB)} & \textnormal{Paar von Typen Tupel} \\
    \textsf{TypeA -> TypeB} & \textnormal{Typenfunktion}
\end{array}
\]
\subsection{Haskell Funktion}
\begin{verbatim}
    f :: X -> Y
\end{verbatim}
\subsubsection{Beispiel}
\begin{verbatim}
    sqrt :: Float -> Float
    first :: (String, Int) -> String
    second :: (String, Int) -> Int
    not :: Bool -> Bool
    and :: [Bool] -> Bool
    logBase :: Float -> Float -> Float
\end{verbatim}
\subsubsection{Anwendung}
So ruft man die Funktion auf:
\begin{verbatim}
    f x
\end{verbatim}
\subsubsection{Funktionsaufbau sich ansehen}
\begin{verbatim}
    :t Funktionsname
\end{verbatim}
\subsection{Operatoren}
\begin{verbatim}
    (op) :: X -> Y -> Z
\end{verbatim}
\subsubsection{Schreibweise}
\begin{verbatim}
    (op) x y 
    x op y 
\end{verbatim}
\subsection{Funktion definieren}
\begin{verbatim}
    f :: Int -> Int
    f x = x * x + x
\end{verbatim}
ist dasselbe wie:
\begin{align*}
    f : \mathbb{Z} \rightarrow \mathbb{Z} \\
    f(x) = x^2 + x
\end{align*}
\subsection{Datei Laden}
\begin{verbatim}
    :l filename
\end{verbatim}

\section{Haskell Pattern Matching}
\subsection{Verschiedene Pattern}
\begin{itemize}
    \item jedes einzelnte Pattern ist einzelnt
    \item es kann mehrere Patterns angegeben werden
    \item von oben nach unten
\end{itemize}
\subsubsection{Beispiel}
\begin{verbatim}
    -- negation
    neg :: Bool -> Bool
    neg False = True

    -- Konjunktion
    (<&>) :: Bool -> Bool -> Bool
    (<&>) True True = True
    (<&>) _ _ = False
\end{verbatim}

\subsection{Unterstriche}
Kann genutzt werdeb un irgendeine Random variable anzugeben.


\section{Haskell Alternativen}

\subsection{If - Else}
\begin{align*}
    \text{abs}(x) = \begin{cases}
        -x & x < 0 \\
        x & \text{sonst}
    \end{cases}
\end{align*}
\subsubsection{Erste Art das umzusetzen}
\begin{verbatim}
    absolute :: Int -> Int
    absolute x = if x < 0 then -x else x
\end{verbatim}
\subsubsection{Mehrere Conditions}
\begin{verbatim}
    absolute :: Int -> Int
    absolute x
        | x < 0 = -x
        | otherwise = x
\end{verbatim}

\section{Haskell Rekursion}

\subsection{Beispiel}
\begin{align*}
    \text{heron}_n : \R^+ \rightarrow \R^+ \\
    \text{heron}(n,a) = \begin{cases}
        (\text{heron}(n - 1, a) + a/\text{heron}(n-1,a))/2 & n > 0 \\
        a & \text{sonst}
    \end{cases}
\end{align*}
\begin{verbatim}
    heronA :: (Int, Double) -> Double
    heronA n a
        | n > 0 = (heronA(n - 1, a) + a / heronA(n - 1, a))/2
        | otherwise = a
\end{verbatim}

\subsection{where - Operator}
Haskell bietet nicht die Möglichkeit Speicherplatz zu reservieren \\
Aber man kann where verwenden für lokale Platzhalter definitionen

\begin{verbatim}
    heronA :: (Int, Double) -> Double
    heronA n a
        | n > 0 = (x + a / x)/2
        | otherwise = a
        where x = heronA(n - 1, a)
\end{verbatim}

\subsection{error-Handling}
\begin{verbatim}
    fibC :: Int -> Int
    fibC 0 = 0
    fibC 1 = 0
    fibC n
        | n < 0 = error "illegal argument"
        | otherwise = fibC(n - 1) + fibC(n - 2)
\end{verbatim}

\subsection{Um -7 in der Konsole zu benutzen beispielsweise}
\begin{verbatim}
    fibC $ -7
\end{verbatim}

\end{document}